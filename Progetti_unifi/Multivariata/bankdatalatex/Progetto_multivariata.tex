
% Default to the notebook output style
\usepackage{adjustbox.sty}
    
sssddsssdss

% Inherit from the specified cell style.




    
\documentclass[11pt]{article}

    
    
    \usepackage[T1]{fontenc}
    % Nicer default font (+ math font) than Computer Modern for most use cases
    \usepackage{mathpazo}

    % Basic figure setup, for now with no caption control since it's done
    % automatically by Pandoc (which extracts ![](path) syntax from Markdown).
    \usepackage{graphicx}
    % We will generate all images so they have a width \maxwidth. This means
    % that they will get their normal width if they fit onto the page, but
    % are scaled down if they would overflow the margins.
    \makeatletter
    \def\maxwidth{\ifdim\Gin@nat@width>\linewidth\linewidth
    \else\Gin@nat@width\fi}
    \makeatother
    \let\Oldincludegraphics\includegraphics
    % Set max figure width to be 80% of text width, for now hardcoded.
    \renewcommand{\includegraphics}[1]{\Oldincludegraphics[width=.8\maxwidth]{#1}}
    % Ensure that by default, figures have no caption (until we provide a
    % proper Figure object with a Caption API and a way to capture that
    % in the conversion process - todo).
    \usepackage{caption}
    \DeclareCaptionLabelFormat{nolabel}{}
    \captionsetup{labelformat=nolabel}

    \usepackage{adjustbox} % Used to constrain images to a maximum size 
    \usepackage{xcolor} % Allow colors to be defined
    \usepackage{enumerate} % Needed for markdown enumerations to work
    \usepackage{geometry} % Used to adjust the document margins
    \usepackage{amsmath} % Equations
    \usepackage{amssymb} % Equations
    \usepackage{textcomp} % defines textquotesingle
    % Hack from http://tex.stackexchange.com/a/47451/13684:
    \AtBeginDocument{%
        \def\PYZsq{\textquotesingle}% Upright quotes in Pygmentized code
    }
    \usepackage{upquote} % Upright quotes for verbatim code
    \usepackage{eurosym} % defines \euro
    \usepackage[mathletters]{ucs} % Extended unicode (utf-8) support
    \usepackage[utf8x]{inputenc} % Allow utf-8 characters in the tex document
    \usepackage{fancyvrb} % verbatim replacement that allows latex
    \usepackage{grffile} % extends the file name processing of package graphics 
                         % to support a larger range 
    % The hyperref package gives us a pdf with properly built
    % internal navigation ('pdf bookmarks' for the table of contents,
    % internal cross-reference links, web links for URLs, etc.)
    \usepackage{hyperref}
    \usepackage{longtable} % longtable support required by pandoc >1.10
    \usepackage{booktabs}  % table support for pandoc > 1.12.2
    \usepackage[inline]{enumitem} % IRkernel/repr support (it uses the enumerate* environment)
    \usepackage[normalem]{ulem} % ulem is needed to support strikethroughs (\sout)
                                % normalem makes italics be italics, not underlines
    

    
    
    % Colors for the hyperref package
    \definecolor{urlcolor}{rgb}{0,.145,.698}
    \definecolor{linkcolor}{rgb}{.71,0.21,0.01}
    \definecolor{citecolor}{rgb}{.12,.54,.11}

    % ANSI colors
    \definecolor{ansi-black}{HTML}{3E424D}
    \definecolor{ansi-black-intense}{HTML}{282C36}
    \definecolor{ansi-red}{HTML}{E75C58}
    \definecolor{ansi-red-intense}{HTML}{B22B31}
    \definecolor{ansi-green}{HTML}{00A250}
    \definecolor{ansi-green-intense}{HTML}{007427}
    \definecolor{ansi-yellow}{HTML}{DDB62B}
    \definecolor{ansi-yellow-intense}{HTML}{B27D12}
    \definecolor{ansi-blue}{HTML}{208FFB}
    \definecolor{ansi-blue-intense}{HTML}{0065CA}
    \definecolor{ansi-magenta}{HTML}{D160C4}
    \definecolor{ansi-magenta-intense}{HTML}{A03196}
    \definecolor{ansi-cyan}{HTML}{60C6C8}
    \definecolor{ansi-cyan-intense}{HTML}{258F8F}
    \definecolor{ansi-white}{HTML}{C5C1B4}
    \definecolor{ansi-white-intense}{HTML}{A1A6B2}

    % commands and environments needed by pandoc snippets
    % extracted from the output of `pandoc -s`
    \providecommand{\tightlist}{%
      \setlength{\itemsep}{0pt}\setlength{\parskip}{0pt}}
    \DefineVerbatimEnvironment{Highlighting}{Verbatim}{commandchars=\\\{\}}
    % Add ',fontsize=\small' for more characters per line
    \newenvironment{Shaded}{}{}
    \newcommand{\KeywordTok}[1]{\textcolor[rgb]{0.00,0.44,0.13}{\textbf{{#1}}}}
    \newcommand{\DataTypeTok}[1]{\textcolor[rgb]{0.56,0.13,0.00}{{#1}}}
    \newcommand{\DecValTok}[1]{\textcolor[rgb]{0.25,0.63,0.44}{{#1}}}
    \newcommand{\BaseNTok}[1]{\textcolor[rgb]{0.25,0.63,0.44}{{#1}}}
    \newcommand{\FloatTok}[1]{\textcolor[rgb]{0.25,0.63,0.44}{{#1}}}
    \newcommand{\CharTok}[1]{\textcolor[rgb]{0.25,0.44,0.63}{{#1}}}
    \newcommand{\StringTok}[1]{\textcolor[rgb]{0.25,0.44,0.63}{{#1}}}
    \newcommand{\CommentTok}[1]{\textcolor[rgb]{0.38,0.63,0.69}{\textit{{#1}}}}
    \newcommand{\OtherTok}[1]{\textcolor[rgb]{0.00,0.44,0.13}{{#1}}}
    \newcommand{\AlertTok}[1]{\textcolor[rgb]{1.00,0.00,0.00}{\textbf{{#1}}}}
    \newcommand{\FunctionTok}[1]{\textcolor[rgb]{0.02,0.16,0.49}{{#1}}}
    \newcommand{\RegionMarkerTok}[1]{{#1}}
    \newcommand{\ErrorTok}[1]{\textcolor[rgb]{1.00,0.00,0.00}{\textbf{{#1}}}}
    \newcommand{\NormalTok}[1]{{#1}}
    
    % Additional commands for more recent versions of Pandoc
    \newcommand{\ConstantTok}[1]{\textcolor[rgb]{0.53,0.00,0.00}{{#1}}}
    \newcommand{\SpecialCharTok}[1]{\textcolor[rgb]{0.25,0.44,0.63}{{#1}}}
    \newcommand{\VerbatimStringTok}[1]{\textcolor[rgb]{0.25,0.44,0.63}{{#1}}}
    \newcommand{\SpecialStringTok}[1]{\textcolor[rgb]{0.73,0.40,0.53}{{#1}}}
    \newcommand{\ImportTok}[1]{{#1}}
    \newcommand{\DocumentationTok}[1]{\textcolor[rgb]{0.73,0.13,0.13}{\textit{{#1}}}}
    \newcommand{\AnnotationTok}[1]{\textcolor[rgb]{0.38,0.63,0.69}{\textbf{\textit{{#1}}}}}
    \newcommand{\CommentVarTok}[1]{\textcolor[rgb]{0.38,0.63,0.69}{\textbf{\textit{{#1}}}}}
    \newcommand{\VariableTok}[1]{\textcolor[rgb]{0.10,0.09,0.49}{{#1}}}
    \newcommand{\ControlFlowTok}[1]{\textcolor[rgb]{0.00,0.44,0.13}{\textbf{{#1}}}}
    \newcommand{\OperatorTok}[1]{\textcolor[rgb]{0.40,0.40,0.40}{{#1}}}
    \newcommand{\BuiltInTok}[1]{{#1}}
    \newcommand{\ExtensionTok}[1]{{#1}}
    \newcommand{\PreprocessorTok}[1]{\textcolor[rgb]{0.74,0.48,0.00}{{#1}}}
    \newcommand{\AttributeTok}[1]{\textcolor[rgb]{0.49,0.56,0.16}{{#1}}}
    \newcommand{\InformationTok}[1]{\textcolor[rgb]{0.38,0.63,0.69}{\textbf{\textit{{#1}}}}}
    \newcommand{\WarningTok}[1]{\textcolor[rgb]{0.38,0.63,0.69}{\textbf{\textit{{#1}}}}}
    
    
    % Define a nice break command that doesn't care if a line doesn't already
    % exist.
    \def\br{\hspace*{\fill} \\* }
    % Math Jax compatability definitions
    \def\gt{>}
    \def\lt{<}
    % Document parameters
    \title{Progetto multivariata}
    
    
    

    % Pygments definitions
    
\makeatletter
\def\PY@reset{\let\PY@it=\relax \let\PY@bf=\relax%
    \let\PY@ul=\relax \let\PY@tc=\relax%
    \let\PY@bc=\relax \let\PY@ff=\relax}
\def\PY@tok#1{\csname PY@tok@#1\endcsname}
\def\PY@toks#1+{\ifx\relax#1\empty\else%
    \PY@tok{#1}\expandafter\PY@toks\fi}
\def\PY@do#1{\PY@bc{\PY@tc{\PY@ul{%
    \PY@it{\PY@bf{\PY@ff{#1}}}}}}}
\def\PY#1#2{\PY@reset\PY@toks#1+\relax+\PY@do{#2}}

\expandafter\def\csname PY@tok@w\endcsname{\def\PY@tc##1{\textcolor[rgb]{0.73,0.73,0.73}{##1}}}
\expandafter\def\csname PY@tok@c\endcsname{\let\PY@it=\textit\def\PY@tc##1{\textcolor[rgb]{0.25,0.50,0.50}{##1}}}
\expandafter\def\csname PY@tok@cp\endcsname{\def\PY@tc##1{\textcolor[rgb]{0.74,0.48,0.00}{##1}}}
\expandafter\def\csname PY@tok@k\endcsname{\let\PY@bf=\textbf\def\PY@tc##1{\textcolor[rgb]{0.00,0.50,0.00}{##1}}}
\expandafter\def\csname PY@tok@kp\endcsname{\def\PY@tc##1{\textcolor[rgb]{0.00,0.50,0.00}{##1}}}
\expandafter\def\csname PY@tok@kt\endcsname{\def\PY@tc##1{\textcolor[rgb]{0.69,0.00,0.25}{##1}}}
\expandafter\def\csname PY@tok@o\endcsname{\def\PY@tc##1{\textcolor[rgb]{0.40,0.40,0.40}{##1}}}
\expandafter\def\csname PY@tok@ow\endcsname{\let\PY@bf=\textbf\def\PY@tc##1{\textcolor[rgb]{0.67,0.13,1.00}{##1}}}
\expandafter\def\csname PY@tok@nb\endcsname{\def\PY@tc##1{\textcolor[rgb]{0.00,0.50,0.00}{##1}}}
\expandafter\def\csname PY@tok@nf\endcsname{\def\PY@tc##1{\textcolor[rgb]{0.00,0.00,1.00}{##1}}}
\expandafter\def\csname PY@tok@nc\endcsname{\let\PY@bf=\textbf\def\PY@tc##1{\textcolor[rgb]{0.00,0.00,1.00}{##1}}}
\expandafter\def\csname PY@tok@nn\endcsname{\let\PY@bf=\textbf\def\PY@tc##1{\textcolor[rgb]{0.00,0.00,1.00}{##1}}}
\expandafter\def\csname PY@tok@ne\endcsname{\let\PY@bf=\textbf\def\PY@tc##1{\textcolor[rgb]{0.82,0.25,0.23}{##1}}}
\expandafter\def\csname PY@tok@nv\endcsname{\def\PY@tc##1{\textcolor[rgb]{0.10,0.09,0.49}{##1}}}
\expandafter\def\csname PY@tok@no\endcsname{\def\PY@tc##1{\textcolor[rgb]{0.53,0.00,0.00}{##1}}}
\expandafter\def\csname PY@tok@nl\endcsname{\def\PY@tc##1{\textcolor[rgb]{0.63,0.63,0.00}{##1}}}
\expandafter\def\csname PY@tok@ni\endcsname{\let\PY@bf=\textbf\def\PY@tc##1{\textcolor[rgb]{0.60,0.60,0.60}{##1}}}
\expandafter\def\csname PY@tok@na\endcsname{\def\PY@tc##1{\textcolor[rgb]{0.49,0.56,0.16}{##1}}}
\expandafter\def\csname PY@tok@nt\endcsname{\let\PY@bf=\textbf\def\PY@tc##1{\textcolor[rgb]{0.00,0.50,0.00}{##1}}}
\expandafter\def\csname PY@tok@nd\endcsname{\def\PY@tc##1{\textcolor[rgb]{0.67,0.13,1.00}{##1}}}
\expandafter\def\csname PY@tok@s\endcsname{\def\PY@tc##1{\textcolor[rgb]{0.73,0.13,0.13}{##1}}}
\expandafter\def\csname PY@tok@sd\endcsname{\let\PY@it=\textit\def\PY@tc##1{\textcolor[rgb]{0.73,0.13,0.13}{##1}}}
\expandafter\def\csname PY@tok@si\endcsname{\let\PY@bf=\textbf\def\PY@tc##1{\textcolor[rgb]{0.73,0.40,0.53}{##1}}}
\expandafter\def\csname PY@tok@se\endcsname{\let\PY@bf=\textbf\def\PY@tc##1{\textcolor[rgb]{0.73,0.40,0.13}{##1}}}
\expandafter\def\csname PY@tok@sr\endcsname{\def\PY@tc##1{\textcolor[rgb]{0.73,0.40,0.53}{##1}}}
\expandafter\def\csname PY@tok@ss\endcsname{\def\PY@tc##1{\textcolor[rgb]{0.10,0.09,0.49}{##1}}}
\expandafter\def\csname PY@tok@sx\endcsname{\def\PY@tc##1{\textcolor[rgb]{0.00,0.50,0.00}{##1}}}
\expandafter\def\csname PY@tok@m\endcsname{\def\PY@tc##1{\textcolor[rgb]{0.40,0.40,0.40}{##1}}}
\expandafter\def\csname PY@tok@gh\endcsname{\let\PY@bf=\textbf\def\PY@tc##1{\textcolor[rgb]{0.00,0.00,0.50}{##1}}}
\expandafter\def\csname PY@tok@gu\endcsname{\let\PY@bf=\textbf\def\PY@tc##1{\textcolor[rgb]{0.50,0.00,0.50}{##1}}}
\expandafter\def\csname PY@tok@gd\endcsname{\def\PY@tc##1{\textcolor[rgb]{0.63,0.00,0.00}{##1}}}
\expandafter\def\csname PY@tok@gi\endcsname{\def\PY@tc##1{\textcolor[rgb]{0.00,0.63,0.00}{##1}}}
\expandafter\def\csname PY@tok@gr\endcsname{\def\PY@tc##1{\textcolor[rgb]{1.00,0.00,0.00}{##1}}}
\expandafter\def\csname PY@tok@ge\endcsname{\let\PY@it=\textit}
\expandafter\def\csname PY@tok@gs\endcsname{\let\PY@bf=\textbf}
\expandafter\def\csname PY@tok@gp\endcsname{\let\PY@bf=\textbf\def\PY@tc##1{\textcolor[rgb]{0.00,0.00,0.50}{##1}}}
\expandafter\def\csname PY@tok@go\endcsname{\def\PY@tc##1{\textcolor[rgb]{0.53,0.53,0.53}{##1}}}
\expandafter\def\csname PY@tok@gt\endcsname{\def\PY@tc##1{\textcolor[rgb]{0.00,0.27,0.87}{##1}}}
\expandafter\def\csname PY@tok@err\endcsname{\def\PY@bc##1{\setlength{\fboxsep}{0pt}\fcolorbox[rgb]{1.00,0.00,0.00}{1,1,1}{\strut ##1}}}
\expandafter\def\csname PY@tok@kc\endcsname{\let\PY@bf=\textbf\def\PY@tc##1{\textcolor[rgb]{0.00,0.50,0.00}{##1}}}
\expandafter\def\csname PY@tok@kd\endcsname{\let\PY@bf=\textbf\def\PY@tc##1{\textcolor[rgb]{0.00,0.50,0.00}{##1}}}
\expandafter\def\csname PY@tok@kn\endcsname{\let\PY@bf=\textbf\def\PY@tc##1{\textcolor[rgb]{0.00,0.50,0.00}{##1}}}
\expandafter\def\csname PY@tok@kr\endcsname{\let\PY@bf=\textbf\def\PY@tc##1{\textcolor[rgb]{0.00,0.50,0.00}{##1}}}
\expandafter\def\csname PY@tok@bp\endcsname{\def\PY@tc##1{\textcolor[rgb]{0.00,0.50,0.00}{##1}}}
\expandafter\def\csname PY@tok@fm\endcsname{\def\PY@tc##1{\textcolor[rgb]{0.00,0.00,1.00}{##1}}}
\expandafter\def\csname PY@tok@vc\endcsname{\def\PY@tc##1{\textcolor[rgb]{0.10,0.09,0.49}{##1}}}
\expandafter\def\csname PY@tok@vg\endcsname{\def\PY@tc##1{\textcolor[rgb]{0.10,0.09,0.49}{##1}}}
\expandafter\def\csname PY@tok@vi\endcsname{\def\PY@tc##1{\textcolor[rgb]{0.10,0.09,0.49}{##1}}}
\expandafter\def\csname PY@tok@vm\endcsname{\def\PY@tc##1{\textcolor[rgb]{0.10,0.09,0.49}{##1}}}
\expandafter\def\csname PY@tok@sa\endcsname{\def\PY@tc##1{\textcolor[rgb]{0.73,0.13,0.13}{##1}}}
\expandafter\def\csname PY@tok@sb\endcsname{\def\PY@tc##1{\textcolor[rgb]{0.73,0.13,0.13}{##1}}}
\expandafter\def\csname PY@tok@sc\endcsname{\def\PY@tc##1{\textcolor[rgb]{0.73,0.13,0.13}{##1}}}
\expandafter\def\csname PY@tok@dl\endcsname{\def\PY@tc##1{\textcolor[rgb]{0.73,0.13,0.13}{##1}}}
\expandafter\def\csname PY@tok@s2\endcsname{\def\PY@tc##1{\textcolor[rgb]{0.73,0.13,0.13}{##1}}}
\expandafter\def\csname PY@tok@sh\endcsname{\def\PY@tc##1{\textcolor[rgb]{0.73,0.13,0.13}{##1}}}
\expandafter\def\csname PY@tok@s1\endcsname{\def\PY@tc##1{\textcolor[rgb]{0.73,0.13,0.13}{##1}}}
\expandafter\def\csname PY@tok@mb\endcsname{\def\PY@tc##1{\textcolor[rgb]{0.40,0.40,0.40}{##1}}}
\expandafter\def\csname PY@tok@mf\endcsname{\def\PY@tc##1{\textcolor[rgb]{0.40,0.40,0.40}{##1}}}
\expandafter\def\csname PY@tok@mh\endcsname{\def\PY@tc##1{\textcolor[rgb]{0.40,0.40,0.40}{##1}}}
\expandafter\def\csname PY@tok@mi\endcsname{\def\PY@tc##1{\textcolor[rgb]{0.40,0.40,0.40}{##1}}}
\expandafter\def\csname PY@tok@il\endcsname{\def\PY@tc##1{\textcolor[rgb]{0.40,0.40,0.40}{##1}}}
\expandafter\def\csname PY@tok@mo\endcsname{\def\PY@tc##1{\textcolor[rgb]{0.40,0.40,0.40}{##1}}}
\expandafter\def\csname PY@tok@ch\endcsname{\let\PY@it=\textit\def\PY@tc##1{\textcolor[rgb]{0.25,0.50,0.50}{##1}}}
\expandafter\def\csname PY@tok@cm\endcsname{\let\PY@it=\textit\def\PY@tc##1{\textcolor[rgb]{0.25,0.50,0.50}{##1}}}
\expandafter\def\csname PY@tok@cpf\endcsname{\let\PY@it=\textit\def\PY@tc##1{\textcolor[rgb]{0.25,0.50,0.50}{##1}}}
\expandafter\def\csname PY@tok@c1\endcsname{\let\PY@it=\textit\def\PY@tc##1{\textcolor[rgb]{0.25,0.50,0.50}{##1}}}
\expandafter\def\csname PY@tok@cs\endcsname{\let\PY@it=\textit\def\PY@tc##1{\textcolor[rgb]{0.25,0.50,0.50}{##1}}}

\def\PYZbs{\char`\\}
\def\PYZus{\char`\_}
\def\PYZob{\char`\{}
\def\PYZcb{\char`\}}
\def\PYZca{\char`\^}
\def\PYZam{\char`\&}
\def\PYZlt{\char`\<}
\def\PYZgt{\char`\>}
\def\PYZsh{\char`\#}
\def\PYZpc{\char`\%}
\def\PYZdl{\char`\$}
\def\PYZhy{\char`\-}
\def\PYZsq{\char`\'}
\def\PYZdq{\char`\"}
\def\PYZti{\char`\~}
% for compatibility with earlier versions
\def\PYZat{@}
\def\PYZlb{[}
\def\PYZrb{]}
\makeatother


    % Exact colors from NB
    \definecolor{incolor}{rgb}{0.0, 0.0, 0.5}
    \definecolor{outcolor}{rgb}{0.545, 0.0, 0.0}



    
    % Prevent overflowing lines due to hard-to-break entities
    \sloppy 
    % Setup hyperref package
    \hypersetup{
      breaklinks=true,  % so long urls are correctly broken across lines
      colorlinks=true,
      urlcolor=urlcolor,
      linkcolor=linkcolor,
      citecolor=citecolor,
      }
    % Slightly bigger margins than the latex defaults
    
    \geometry{verbose,tmargin=1in,bmargin=1in,lmargin=1in,rmargin=1in}
    
    

    \begin{document}
    
    
    \maketitle
    
    

    
    \begin{Verbatim}[commandchars=\\\{\}]
{\color{incolor}In [{\color{incolor}72}]:} \PY{k+kn}{import} \PY{n+nn}{pandas} \PY{k}{as} \PY{n+nn}{pd}
         \PY{k+kn}{import} \PY{n+nn}{numpy} \PY{k}{as} \PY{n+nn}{np}
         \PY{k+kn}{import} \PY{n+nn}{pandas\PYZus{}profiling}
         \PY{k+kn}{import} \PY{n+nn}{os}
         \PY{c+c1}{\PYZsh{}per modelli statistici}
         \PY{k+kn}{import} \PY{n+nn}{statsmodels}\PY{n+nn}{.}\PY{n+nn}{formula}\PY{n+nn}{.}\PY{n+nn}{api} \PY{k}{as} \PY{n+nn}{sm}
         \PY{c+c1}{\PYZsh{} Let us do the visualization imports:}
         \PY{k+kn}{import} \PY{n+nn}{matplotlib}\PY{n+nn}{.}\PY{n+nn}{pyplot} \PY{k}{as} \PY{n+nn}{plt}
         \PY{k+kn}{import} \PY{n+nn}{seaborn} \PY{k}{as} \PY{n+nn}{sns}
         \PY{o}{\PYZpc{}}\PY{k}{matplotlib} inline
         \PY{c+c1}{\PYZsh{} carico il dataset parziale}
         \PY{n}{df} \PY{o}{=} \PY{n}{pd}\PY{o}{.}\PY{n}{read\PYZus{}csv}\PY{p}{(}\PY{l+s+s1}{\PYZsq{}}\PY{l+s+s1}{bank.csv}\PY{l+s+s1}{\PYZsq{}}\PY{p}{,} \PY{n}{sep}\PY{o}{=}\PY{l+s+s2}{\PYZdq{}}\PY{l+s+s2}{;}\PY{l+s+s2}{\PYZdq{}}\PY{p}{)}
         \PY{c+c1}{\PYZsh{} df.dtypes}
\end{Verbatim}


    \begin{Verbatim}[commandchars=\\\{\}]
{\color{incolor}In [{\color{incolor}3}]:} \PY{c+c1}{\PYZsh{} pandas\PYZus{}profiling.ProfileReport(df)}
\end{Verbatim}


\begin{Verbatim}[commandchars=\\\{\}]
{\color{outcolor}Out[{\color{outcolor}3}]:} <pandas\_profiling.ProfileReport at 0x7fbdd06e5240>
\end{Verbatim}
            
    Input variables: \# bank client data: 1 - age (numeric) 2 - job : type
of job (categorical:
"admin.","unknown","unemployed","management","housemaid","entrepreneur","student",
"blue-collar","self-employed","retired","technician","services") 3 -
marital : marital status (categorical: "married","divorced","single";
note: "divorced" means divorced or widowed) 4 - education (categorical:
"unknown","secondary","primary","tertiary") 5 - default: has credit in
default? (binary: "yes","no") 6 - balance: average yearly balance, in
euros (numeric) 7 - housing: has housing loan? (binary: "yes","no") 8 -
loan: has personal loan? (binary: "yes","no") \# related with the last
contact of the current campaign: 9 - contact: contact communication type
(categorical: "unknown","telephone","cellular") 10 - day: last contact
day of the month (numeric) 11 - month: last contact month of year
(categorical: "jan", "feb", "mar", ..., "nov", "dec") 12 - duration:
last contact duration, in seconds (numeric) \# other attributes: 13 -
campaign: number of contacts performed during this campaign and for this
client (numeric, includes last contact) 14 - pdays: number of days that
passed by after the client was last contacted from a previous campaign
(numeric, -1 means client was not previously contacted) 15 - previous:
number of contacts performed before this campaign and for this client
(numeric) 16 - poutcome: outcome of the previous marketing campaign
(categorical: "unknown","other","failure","success")

Output variable (desired target): 17 - y - has the client subscribed a
term deposit? (binary: "yes","no")

\begin{enumerate}
\def\labelenumi{\arabic{enumi}.}
\setcounter{enumi}{7}
\tightlist
\item
  Missing Attribute Values: None
\end{enumerate}

    \# 0) PREPROCESSING:

    \begin{Verbatim}[commandchars=\\\{\}]
{\color{incolor}In [{\color{incolor}73}]:} \PY{c+c1}{\PYZsh{} one\PYZus{}hot\PYZus{}encoding = pd.get\PYZus{}dummies(df)}
         \PY{n}{df}\PY{o}{.}\PY{n}{head}\PY{p}{(}\PY{p}{)}
         
         \PY{c+c1}{\PYZsh{} df.poutcome.value\PYZus{}counts()}
\end{Verbatim}


\begin{Verbatim}[commandchars=\\\{\}]
{\color{outcolor}Out[{\color{outcolor}73}]:}    age          job  marital  education default  balance housing loan  \textbackslash{}
         0   30   unemployed  married    primary      no     1787      no   no   
         1   33     services  married  secondary      no     4789     yes  yes   
         2   35   management   single   tertiary      no     1350     yes   no   
         3   30   management  married   tertiary      no     1476     yes  yes   
         4   59  blue-collar  married  secondary      no        0     yes   no   
         
             contact  day month  duration  campaign  pdays  previous poutcome   y  
         0  cellular   19   oct        79         1     -1         0  unknown  no  
         1  cellular   11   may       220         1    339         4  failure  no  
         2  cellular   16   apr       185         1    330         1  failure  no  
         3   unknown    3   jun       199         4     -1         0  unknown  no  
         4   unknown    5   may       226         1     -1         0  unknown  no  
\end{Verbatim}
            
    \begin{Verbatim}[commandchars=\\\{\}]
{\color{incolor}In [{\color{incolor}76}]:} \PY{n}{df}\PY{o}{.}\PY{n}{job}\PY{o}{.}\PY{n}{value\PYZus{}counts}\PY{p}{(}\PY{p}{)}
\end{Verbatim}


\begin{Verbatim}[commandchars=\\\{\}]
{\color{outcolor}Out[{\color{outcolor}76}]:} management       969
         blue-collar      946
         technician       768
         admin.           478
         services         417
         retired          230
         self-employed    183
         entrepreneur     168
         unemployed       128
         housemaid        112
         student           84
         unknown           38
         Name: job, dtype: int64
\end{Verbatim}
            
    \begin{Verbatim}[commandchars=\\\{\}]
{\color{incolor}In [{\color{incolor}68}]:} \PY{n}{var\PYZus{}continue} \PY{o}{=} \PY{p}{[}\PY{l+s+s1}{\PYZsq{}}\PY{l+s+s1}{age}\PY{l+s+s1}{\PYZsq{}}\PY{p}{,} \PY{l+s+s1}{\PYZsq{}}\PY{l+s+s1}{balance}\PY{l+s+s1}{\PYZsq{}}\PY{p}{,} \PY{l+s+s1}{\PYZsq{}}\PY{l+s+s1}{day}\PY{l+s+s1}{\PYZsq{}}\PY{p}{,} \PY{l+s+s1}{\PYZsq{}}\PY{l+s+s1}{duration}\PY{l+s+s1}{\PYZsq{}}\PY{p}{,} \PY{l+s+s1}{\PYZsq{}}\PY{l+s+s1}{campaign}\PY{l+s+s1}{\PYZsq{}}\PY{p}{]}
         \PY{n}{var\PYZus{}temporali} \PY{o}{=} \PY{p}{[}\PY{l+s+s1}{\PYZsq{}}\PY{l+s+s1}{day}\PY{l+s+s1}{\PYZsq{}}\PY{p}{,} \PY{l+s+s1}{\PYZsq{}}\PY{l+s+s1}{month}\PY{l+s+s1}{\PYZsq{}}\PY{p}{]}
         \PY{n}{var\PYZus{}dubbie} \PY{o}{=} \PY{p}{[}\PY{l+s+s1}{\PYZsq{}}\PY{l+s+s1}{pdays}\PY{l+s+s1}{\PYZsq{}}\PY{p}{]}
         \PY{n}{var\PYZus{}dicotomiche} \PY{o}{=} \PY{p}{[}\PY{l+s+s1}{\PYZsq{}}\PY{l+s+s1}{defaut}\PY{l+s+s1}{\PYZsq{}}\PY{p}{,} \PY{l+s+s1}{\PYZsq{}}\PY{l+s+s1}{housing}\PY{l+s+s1}{\PYZsq{}}\PY{p}{,} \PY{l+s+s1}{\PYZsq{}}\PY{l+s+s1}{loan}\PY{l+s+s1}{\PYZsq{}}\PY{p}{]}
         \PY{n}{var\PYZus{}categoriche} \PY{o}{=} \PY{p}{[}\PY{l+s+s1}{\PYZsq{}}\PY{l+s+s1}{job}\PY{l+s+s1}{\PYZsq{}}\PY{p}{,}\PY{l+s+s1}{\PYZsq{}}\PY{l+s+s1}{marital}\PY{l+s+s1}{\PYZsq{}}\PY{p}{]}
\end{Verbatim}


    \begin{Verbatim}[commandchars=\\\{\}]
{\color{incolor}In [{\color{incolor}79}]:} \PY{n}{one\PYZus{}hot\PYZus{}encoding} \PY{o}{=} \PY{n}{pd}\PY{o}{.}\PY{n}{get\PYZus{}dummies}\PY{p}{(}\PY{n}{df}\PY{p}{)}
         \PY{n}{grandezze} \PY{o}{=} \PY{p}{(}\PY{n}{one\PYZus{}hot\PYZus{}encoding}\PY{o}{.}\PY{n}{shape}\PY{p}{,} \PY{n}{df}\PY{o}{.}\PY{n}{shape}\PY{p}{)}
         \PY{n}{grandezze}
\end{Verbatim}


\begin{Verbatim}[commandchars=\\\{\}]
{\color{outcolor}Out[{\color{outcolor}79}]:} ((4521, 53), (4521, 17))
\end{Verbatim}
            
    \begin{Verbatim}[commandchars=\\\{\}]
{\color{incolor}In [{\color{incolor}77}]:} \PY{c+c1}{\PYZsh{} dfcovariates = ([\PYZsq{}age\PYZsq{}, \PYZsq{}job\PYZsq{}, \PYZsq{}marital\PYZsq{}, \PYZsq{}education\PYZsq{}, \PYZsq{}default\PYZsq{}, \PYZsq{}balance\PYZsq{}, \PYZsq{}housing\PYZsq{},}
         \PY{c+c1}{\PYZsh{}        \PYZsq{}loan\PYZsq{}, \PYZsq{}contact\PYZsq{}, \PYZsq{}day\PYZsq{}, \PYZsq{}month\PYZsq{}, \PYZsq{}duration\PYZsq{}, \PYZsq{}campaign\PYZsq{}, \PYZsq{}pdays\PYZsq{},}
         \PY{c+c1}{\PYZsh{}        \PYZsq{}previous\PYZsq{}, \PYZsq{}poutcome\PYZsq{}])}
         \PY{c+c1}{\PYZsh{} X = one\PYZus{}hot\PYZus{}encoding[dfcovariates]}
         \PY{c+c1}{\PYZsh{} y = one\PYZus{}hot\PYZus{}encoding.y}
         
         \PY{c+c1}{\PYZsh{} model = sm.Logit(y, X)}
         \PY{c+c1}{\PYZsh{} result = model.fit()}
\end{Verbatim}


    \section{VOGLIO VEDERE L'EFFETTO DEL TIPO DI LAVORO
SULL'OUTCOME:}\label{voglio-vedere-leffetto-del-tipo-di-lavoro-sulloutcome}

    \begin{Verbatim}[commandchars=\\\{\}]
{\color{incolor}In [{\color{incolor}135}]:} \PY{k+kn}{import} \PY{n+nn}{statsmodels}\PY{n+nn}{.}\PY{n+nn}{api} \PY{k}{as} \PY{n+nn}{sm}
          
          \PY{n}{onehotencoding\PYZus{}y} \PY{o}{=} \PY{n}{pd}\PY{o}{.}\PY{n}{get\PYZus{}dummies}\PY{p}{(}\PY{n}{df}\PY{o}{.}\PY{n}{y}\PY{p}{)}
          \PY{n}{onehotencoding\PYZus{}job} \PY{o}{=} \PY{n}{pd}\PY{o}{.}\PY{n}{get\PYZus{}dummies}\PY{p}{(}\PY{n}{df}\PY{o}{.}\PY{n}{job}\PY{p}{)}
          
          \PY{n}{y} \PY{o}{=} \PY{n}{pd}\PY{o}{.}\PY{n}{Series}\PY{o}{.}\PY{n}{to\PYZus{}frame}\PY{p}{(}\PY{n}{onehotencoding\PYZus{}y}\PY{o}{.}\PY{n}{yes}\PY{p}{)} \PY{c+c1}{\PYZsh{}hanno stipulato il contratto}
          
          \PY{n}{X} \PY{o}{=} \PY{n}{onehotencoding\PYZus{}job}
          \PY{n+nb}{type}\PY{p}{(}\PY{n}{y}\PY{p}{)}
          
          \PY{c+c1}{\PYZsh{}LINEAR MODEL}
          \PY{n+nb}{print}\PY{p}{(}\PY{l+s+s2}{\PYZdq{}}\PY{l+s+s2}{LINEAR MODEL}\PY{l+s+s2}{\PYZdq{}}\PY{p}{)}
          \PY{c+c1}{\PYZsh{}X = sm.add\PYZus{}constant(X)}
          \PY{n}{est} \PY{o}{=} \PY{n}{sm}\PY{o}{.}\PY{n}{OLS}\PY{p}{(}\PY{n}{y}\PY{p}{,} \PY{n}{X}\PY{p}{)}
          \PY{n}{est\PYZus{}linear} \PY{o}{=} \PY{n}{est}\PY{o}{.}\PY{n}{fit}\PY{p}{(}\PY{p}{)}
          \PY{n+nb}{print}\PY{p}{(}\PY{n}{est\PYZus{}linear}\PY{o}{.}\PY{n}{summary}\PY{p}{(}\PY{p}{)}\PY{p}{)}
          
          \PY{c+c1}{\PYZsh{}LOGIT MODEL}
          \PY{n+nb}{print}\PY{p}{(}\PY{l+s+s2}{\PYZdq{}}\PY{l+s+s2}{LOGIT MODEL}\PY{l+s+s2}{\PYZdq{}}\PY{p}{)}
          \PY{n}{est2} \PY{o}{=} \PY{n}{sm}\PY{o}{.}\PY{n}{Logit}\PY{p}{(}\PY{n}{y}\PY{p}{,}\PY{n}{X}\PY{p}{)}
          \PY{n}{est\PYZus{}logit} \PY{o}{=} \PY{n}{est2}\PY{o}{.}\PY{n}{fit}\PY{p}{(}\PY{p}{)}
          \PY{n+nb}{print}\PY{p}{(}\PY{n}{est\PYZus{}logit}\PY{o}{.}\PY{n}{summary}\PY{p}{(}\PY{p}{)}\PY{p}{)}
          \PY{n+nb}{print}\PY{p}{(}\PY{l+s+s2}{\PYZdq{}}\PY{l+s+s2}{OR}\PY{l+s+s2}{\PYZdq{}}\PY{p}{)}
          \PY{n}{OR} \PY{o}{=} \PY{n}{np}\PY{o}{.}\PY{n}{exp}\PY{p}{(}\PY{n}{est\PYZus{}logit}\PY{o}{.}\PY{n}{params}\PY{p}{)}
          \PY{n+nb}{print}\PY{p}{(}\PY{n}{OR}\PY{p}{)}
          \PY{c+c1}{\PYZsh{}EXSTRACT PROBABILITY}
          \PY{n+nb}{print}\PY{p}{(}\PY{l+s+s2}{\PYZdq{}}\PY{l+s+s2}{PROBABILITIES}\PY{l+s+s2}{\PYZdq{}}\PY{p}{)}
          \PY{n}{probabilities} \PY{o}{=} \PY{n}{OR}\PY{o}{/}\PY{p}{(}\PY{l+m+mi}{1}\PY{o}{+}\PY{n}{OR}\PY{p}{)}
          \PY{n+nb}{print}\PY{p}{(}\PY{n}{probabilities}\PY{p}{)}
\end{Verbatim}


    \begin{Verbatim}[commandchars=\\\{\}]
LINEAR MODEL
                            OLS Regression Results                            
==============================================================================
Dep. Variable:                    yes   R-squared:                       0.015
Model:                            OLS   Adj. R-squared:                  0.013
Method:                 Least Squares   F-statistic:                     6.352
Date:                Mon, 13 May 2019   Prob (F-statistic):           1.59e-10
Time:                        17:21:37   Log-Likelihood:                -1219.1
No. Observations:                4521   AIC:                             2462.
Df Residuals:                    4509   BIC:                             2539.
Df Model:                          11                                         
Covariance Type:            nonrobust                                         
=================================================================================
                    coef    std err          t      P>|t|      [0.025      0.975]
---------------------------------------------------------------------------------
admin.            0.1213      0.015      8.361      0.000       0.093       0.150
blue-collar       0.0729      0.010      7.071      0.000       0.053       0.093
entrepreneur      0.0893      0.024      3.647      0.000       0.041       0.137
housemaid         0.1250      0.030      4.169      0.000       0.066       0.184
management        0.1352      0.010     13.263      0.000       0.115       0.155
retired           0.2348      0.021     11.222      0.000       0.194       0.276
self-employed     0.1093      0.023      4.660      0.000       0.063       0.155
services          0.0911      0.016      5.865      0.000       0.061       0.122
student           0.2262      0.035      6.534      0.000       0.158       0.294
technician        0.1081      0.011      9.439      0.000       0.086       0.131
unemployed        0.1016      0.028      3.621      0.000       0.047       0.157
unknown           0.1842      0.051      3.579      0.000       0.083       0.285
==============================================================================
Omnibus:                     2000.632   Durbin-Watson:                   1.944
Prob(Omnibus):                  0.000   Jarque-Bera (JB):             6812.044
Skew:                           2.359   Prob(JB):                         0.00
Kurtosis:                       6.727   Cond. No.                         5.05
==============================================================================

Warnings:
[1] Standard Errors assume that the covariance matrix of the errors is correctly specified.
LOGIT MODEL
Optimization terminated successfully.
         Current function value: 474.400369
         Iterations 6
                           Logit Regression Results                           
==============================================================================
Dep. Variable:                    yes   No. Observations:                 4521
Model:                          Logit   Df Residuals:                     4509
Method:                           MLE   Df Model:                           11
Date:                Mon, 13 May 2019   Pseudo R-squ.:                     inf
Time:                        17:21:37   Log-Likelihood:            -2.1448e+06
converged:                       True   LL-Null:                        0.0000
                                        LLR p-value:                     1.000
=================================================================================
                    coef    std err          z      P>|z|      [0.025      0.975]
---------------------------------------------------------------------------------
admin.           -1.9798      0.140    -14.133      0.000      -2.254      -1.705
blue-collar      -2.5424      0.125    -20.334      0.000      -2.787      -2.297
entrepreneur     -2.3224      0.271     -8.584      0.000      -2.853      -1.792
housemaid        -1.9459      0.286     -6.811      0.000      -2.506      -1.386
management       -1.8558      0.094    -19.753      0.000      -2.040      -1.672
retired          -1.1815      0.156     -7.595      0.000      -1.486      -0.877
self-employed    -2.0980      0.237     -8.855      0.000      -2.562      -1.634
services         -2.3000      0.170    -13.516      0.000      -2.633      -1.966
student          -1.2299      0.261     -4.716      0.000      -1.741      -0.719
technician       -2.1106      0.116    -18.160      0.000      -2.338      -1.883
unemployed       -2.1800      0.293     -7.450      0.000      -2.753      -1.606
unknown          -1.4881      0.418     -3.556      0.000      -2.308      -0.668
=================================================================================
OR
admin.           0.138095
blue-collar      0.078677
entrepreneur     0.098039
housemaid        0.142857
management       0.156325
retired          0.306818
self-employed    0.122699
services         0.100264
student          0.292308
technician       0.121168
unemployed       0.113043
unknown          0.225806
dtype: float64
PROBABILITIES
admin.           0.121339
blue-collar      0.072939
entrepreneur     0.089286
housemaid        0.125000
management       0.135191
retired          0.234783
self-employed    0.109290
services         0.091127
student          0.226190
technician       0.108073
unemployed       0.101563
unknown          0.184211
dtype: float64

    \end{Verbatim}

    \begin{Verbatim}[commandchars=\\\{\}]
/home/dami/anaconda/lib/python3.6/site-packages/statsmodels/base/model.py:488: HessianInversionWarning: Inverting hessian failed, no bse or cov\_params available
  'available', HessianInversionWarning)
/home/dami/anaconda/lib/python3.6/site-packages/statsmodels/base/model.py:488: HessianInversionWarning: Inverting hessian failed, no bse or cov\_params available
  'available', HessianInversionWarning)
/home/dami/anaconda/lib/python3.6/site-packages/statsmodels/discrete/discrete\_model.py:3313: RuntimeWarning: divide by zero encountered in double\_scalars
  return 1 - self.llf/self.llnull

    \end{Verbatim}

    \begin{Verbatim}[commandchars=\\\{\}]
{\color{incolor}In [{\color{incolor}185}]:} \PY{n}{df}\PY{o}{.}\PY{n}{job}\PY{o}{.}\PY{n}{value\PYZus{}counts}\PY{p}{(}\PY{p}{)}
          
          \PY{n}{df}\PY{o}{.}\PY{n}{groupby}\PY{p}{(}\PY{p}{[}\PY{l+s+s1}{\PYZsq{}}\PY{l+s+s1}{job}\PY{l+s+s1}{\PYZsq{}}\PY{p}{,} \PY{l+s+s1}{\PYZsq{}}\PY{l+s+s1}{y}\PY{l+s+s1}{\PYZsq{}}\PY{p}{]}\PY{p}{)}\PY{o}{.}\PY{n}{size}\PY{p}{(}\PY{p}{)}
\end{Verbatim}


\begin{Verbatim}[commandchars=\\\{\}]
{\color{outcolor}Out[{\color{outcolor}185}]:} job            y  
          admin.         no     420
                         yes     58
          blue-collar    no     877
                         yes     69
          entrepreneur   no     153
                         yes     15
          housemaid      no      98
                         yes     14
          management     no     838
                         yes    131
          retired        no     176
                         yes     54
          self-employed  no     163
                         yes     20
          services       no     379
                         yes     38
          student        no      65
                         yes     19
          technician     no     685
                         yes     83
          unemployed     no     115
                         yes     13
          unknown        no      31
                         yes      7
          dtype: int64
\end{Verbatim}
            
    \begin{Verbatim}[commandchars=\\\{\}]
{\color{incolor}In [{\color{incolor}188}]:} \PY{l+m+mi}{58}\PY{o}{/}\PY{l+m+mi}{420}
          \PY{l+m+mi}{58}\PY{o}{/}\PY{p}{(}\PY{l+m+mi}{420}\PY{o}{+}\PY{l+m+mi}{58}\PY{p}{)}
\end{Verbatim}


\begin{Verbatim}[commandchars=\\\{\}]
{\color{outcolor}Out[{\color{outcolor}188}]:} 0.12133891213389121
\end{Verbatim}
            
    \section{ADESSO SCELGO LE COVARIATE DA INSERIRE NEL
MODELLO}\label{adesso-scelgo-le-covariate-da-inserire-nel-modello}


    % Add a bibliography block to the postdoc
    
    
    
    \end{document}
